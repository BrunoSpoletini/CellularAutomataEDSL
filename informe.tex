\documentclass[a4paper,11pt]{article}

\usepackage[T1]{fontenc}
\usepackage[utf8]{inputenc}
\usepackage{graphicx}
\usepackage{xcolor}

\renewcommand\familydefault{\sfdefault}
\usepackage{tgheros}

\usepackage[spanish]{babel}
\usepackage{amsmath,amssymb,amsthm,textcomp}
\usepackage{enumerate}
\usepackage{multicol}
\usepackage{tikz}

\usepackage{geometry}
\geometry{left=25mm,right=25mm,%
bindingoffset=0mm, top=20mm,bottom=20mm}


\def\PARTICIPANTES{Spoletini Bruno}
\def\MATERIA{Análisis de Lenguajes de Programación}
\def\TRABAJOPRACTICO{Trabajo Práctico Final}
\def\TITULOTRABAJO{Autómatas celulares}
\def\LEFTNOTE{}


\linespread{1.3}

\newcommand{\linia}{\rule{\linewidth}{0.5pt}}

% custom theorems if needed
\newtheoremstyle{mytheor}
    {1ex}{1ex}{\normalfont}{0pt}{\scshape}{.}{1ex}
    {{\thmname{#1 }}{\thmnumber{#2}}{\thmnote{ (#3)}}}

\theoremstyle{mytheor}
\newtheorem{defi}{Definition}

% my own titles
\makeatletter
\renewcommand{\maketitle}{
\begin{center}
\vspace{2ex}
{\huge \textsc{\@title}}
\vspace{1ex}
\\
\linia\\
\@author \hfill \@date
\vspace{4ex}
\end{center}
}
\makeatother
%%%

\usepackage{hyperref}
% custom footers and headers
\usepackage{fancyhdr}
\pagestyle{fancy}
\lhead{}
\chead{}
\rhead{}
\lfoot{\LEFTNOTE}
\cfoot{}
\rfoot{P\'agina \thepage}
\renewcommand{\headrulewidth}{0pt}
\renewcommand{\footrulewidth}{0pt}
%

% code listing settings
\usepackage{listings}
\lstset{
    language=C,
    basicstyle=\ttfamily\small,
    aboveskip={0.05\baselineskip},
    belowskip={0.1\baselineskip},
    columns=fixed,
    extendedchars=true,
    breaklines=true,
    tabsize=2,
    prebreak=\raisebox{0ex}[0ex][0ex]{\ensuremath{\hookleftarrow}},
    frame=lines,
    showtabs=false,
    showspaces=false,
    showstringspaces=false,
    keywordstyle=\color[rgb]{0.627,0.126,0.941},
    commentstyle=\color[rgb]{0.133,0.545,0.133},
    stringstyle=\color[rgb]{01,0,0},
    numbers=left,
    numberstyle=\small,
    stepnumber=1,
    numbersep=10pt,
    captionpos=t,
    escapeinside={\%*}{*)}
}

%%%----------%%%----------%%%----------%%%----------%%%

\begin{document}
\begin{titlepage}
	\centering

	\begin{figure}[t]
	\raggedleft
    \includegraphics[scale=0.8]{fceia_logo.png}
    \hfill
	\raggedright
    \includegraphics[scale=0.15]{logounr.png}
    \end{figure}
    \vspace{5cm}
	{\scshape\LARGE UNR - FCEIA \par}
	\vspace{1cm}
	{\scshape\Large \MATERIA\par}
	\vspace{1.5cm}
	{\Huge\bfseries \TITULOTRABAJO\par}
	\vspace{2cm}
	{\Large\itshape \PARTICIPANTES \par}
	\vspace{4cm}


	\vfill

\end{titlepage}

\title{ \TITULOTRABAJO}

\author{\PARTICIPANTES}

\date{\today}

\maketitle





\section{Introducción}
 (que hace, por que)
 \subsection{Qué es un autómata celular?}
Un autómata celular (AC) es un modelo matemático y computacional para un sistema dinámico que evoluciona en pasos discretos. Este modelo consiste en una cuadrícula formada por ``Células'' que pueden cambiar de estado dependiendo de las leyes que evalúan los estados de las células vecinas.\\
Los elementos básicos de un AC son:
\begin{itemize}
    \item Cuadrícula de células
    \item Conjunto finito de estados que puede tomar cada célula
    \item Vecindad de cada célula
    \item Función de transición que se aplica a cada célula en cada paso discreto de tiempo.
\end{itemize}

 \subsection{Descripción del proyecto}
Este proyecto consiste en un DSL (Domain Specific Lenguaje) embebido en Haskell, el cual permite definir distintos tipos de células, ubicarlas en una cuardrícula, ejecutar funciones de transición y visualizar la simulación a través de una interfaz gráfica de usuario (GUI). \\
\\
\section{Manual de uso e instalación de software}
Para utilizar el DSL de autómatas celulares, es necesario tener instalado Haskell y algunas bibliotecas adicionales. A continuación, se detallan los pasos para la instalación y uso del software.

\subsection{Instalación de Haskell}
1. Descargue e instale la plataforma Haskell desde \url{https://www.haskell.org/platform/}.
2. Siga las instrucciones de instalación específicas para su sistema operativo.

\subsection{Ejecución del proyecto}
1. Clone el repositorio del proyecto desde GitHub:
\begin{verbatim}
git clone https://github.com/usuario/proyecto-automatas-celulares.git
\end{verbatim}
2. Navegue al directorio del proyecto:
\begin{verbatim}
cd proyecto-automatas-celulares
\end{verbatim}
3. Compile y ejecute el proyecto:
\begin{verbatim}
cabal run
\end{verbatim}

\section{Organización de los archivos}
El proyecto está organizado en varios módulos y archivos para facilitar su mantenimiento y comprensión. A continuación, se describe la estructura del proyecto:

\begin{itemize}
    \item \textbf{Main.hs}: Archivo principal que contiene el punto de entrada del programa.
    \item \textbf{Automata.hs}: Define las estructuras de datos y funciones relacionadas con los autómatas celulares.
    \item \textbf{Grid.hs}: Contiene las funciones para manipular la cuadrícula de células.
    \item \textbf{GUI.hs}: Implementa la interfaz gráfica de usuario para visualizar la simulación.
    \item \textbf{Utils.hs}: Funciones utilitarias que se utilizan en varios módulos del proyecto.
\end{itemize}

\section{Decisiones de diseño}
Durante el desarrollo del proyecto, se tomaron varias decisiones de diseño importantes para asegurar la flexibilidad y extensibilidad del DSL:

\begin{itemize}
    \item \textbf{Modularidad}: El proyecto se dividió en varios módulos para separar las responsabilidades y facilitar el mantenimiento del código.
    \item \textbf{Interfaz gráfica de usuario (GUI)}: Se decidió utilizar la biblioteca Gloss para implementar la GUI debido a su simplicidad y facilidad de uso.
    \item \textbf{Funciones de transición}: Las funciones de transición se definieron de manera que puedan ser fácilmente extendidas para soportar nuevos tipos de células y reglas de evolución.
\end{itemize}

\section{Fuentes consultadas}
Para el desarrollo de este proyecto, se consultaron las siguientes fuentes:

\begin{itemize}
    \item \url{https://es.wikipedia.org/wiki/Autómata_celular}
    \item \url{https://www.haskell.org/}
    \item \url{http://gloss.ouroborus.net/}
\end{itemize}


 
% \section{Manual de uso e instalacion de software}






% \section{Organizacion de los archivos}

%  (modulos y archivos adicionales)






% \section{Decisiones de diseño}

% Decisiones de dise˜no importantes y cualquier otra
% informaci´on que ayude a entender el c´odigo.






% \section{Fuentes consultadas}

% I Bibliograf´ıa, detalle de software de terceros (ya sean
% bibliotecas, copy/paste, o simplemente cdigo que hayan
% usado de inspiracin)

% \url{https://es.wikipedia.org/wiki}\\

% \end{document}

